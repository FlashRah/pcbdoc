
\section{Установка}

Класс \emph{pcbdoc} разрабатывался и использовался на компьютере с операционной
системой \emph{Debian GNU/Linux}. Вероятнее всего, он также будет работать и на
\emph{Windows}, и на \emph{MacOS}.

На машине пользователя \emph{pcbdoc} должен быть установлен и настроен
дистрибутив \emph{TeX~Live}. Процесс установки и настройки \emph{Tex~Live}
для конкретной платформы описан в его официальной документации.

Для установки \emph{pcbdoc} нужно скопировать дерево исходных файлов класса в
директорию, на которую указывает переменная \texttt{TEXMFHOME}, и установить
используемые в \emph{pcbdoc} шрифты для конкретной операционной системы.
\texttt{TEXMFHOME} является переменной дистрибутива \emph{Tex~Live},
указывающей на дерево, которое пользователи \emph{Tex~Live} могут использовать
для установки собственных пакетов, шрифтов или обновлённых версий системных
пакетов. Оно находится в домашней директории, своей для каждого пользователя.

Архив с классом \emph{pcbdoc} содержит две директории:
\begin{itemize}
  \item \coloremph{texmf}

  В этой директории находится директория \coloremph{tex}, дерево исходных файлов
  \emph{pcbdoc}.

  \item \coloremph{.fonts}

  В этой директории находятся файлы используемых в \emph{pcbdoc} шрифтов.

\end{itemize}

Далее подразумевается, что на машине пользователя установлена операционная
система \emph{Debian GNU/Linux}, а командной оболочкой является \emph{bash}.

Переменная \texttt{TEXMFHOME} указывает на директорию \emph{texmf} в домашнем
каталоге пользователя. В этом можно убедиться, выполнив команду

\begin{SaveVerbatim}{InstallCode}
  tlmgr conf | grep TEXMFHOME
\end{SaveVerbatim}
\colorbox{terminalcolor}{\BUseVerbatim{InstallCode}}

Таким образом, установка \emph{pcbdoc} сводится к извлечению содержимого архива
в домашний каталог пользователя:

\begin{SaveVerbatim}{InstallCode}
  unzip pcbdoc.zip -d ~
\end{SaveVerbatim}
\colorbox{terminalcolor}{\BUseVerbatim{InstallCode}}
