
\section{Спецификация}

Исходный код спецификации, как и любого другого документа \LaTeX{}, должен
начинаться с команды \bfemph{\textbackslash{}documentclass}. В основном аргументе
команды(в фигурных скобках) следует указать класс документа \sfemph{pcbdoc}, а в
необязательном(в квадратных скобках) --- ключ \sfemph{doctype=spec}.

\begin{pcbdoccode}
\documentclass[doctype=spec]{pcbdoc}
\end{pcbdoccode}

Далее(в преамбуле) должны находиться команды заполнения полей документа. Их можно
разместить либо непосредственно в преамбуле, либо в отдельном пакете, включаемом в
преамбулу с помощью команды \bfsf{\textbackslash{}usepackage}. Тело документа,
находящееся после преамбулы, должно начинаться с команды

\begin{pcbdoccode}
\begin{document}
\end{pcbdoccode}

а заканчиваться командой

\begin{pcbdoccode}
\end{document}
\end{pcbdoccode}

Внутри тела документа должно находиться
окружение \sfemph{Specification}, которое начинается с команды

\begin{pcbdoccode}
\begin{Specification}
\end{pcbdoccode}

а заканчивается командой

\begin{pcbdoccode}
\end{Specification}
\end{pcbdoccode}

Внутри окружения \sfemph{Specification} должны находиться команды заполнения строк
спецификации, приведённые в \bfsf{Таблице~\ref{tabular:speclines}}.\newpage
