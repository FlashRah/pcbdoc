
\section{Команды заполнения полей документа}

Команды заполнения полей, приведённые в \bfemph{Таблице~\ref{tabular:fields}}, должны
находиться либо непосредственно в преамбуле документа, либо в отдельном пакете,
включаемом в преамбулу с помощью команды \bfemph{\textbackslash{}usepackage}.

\vspace{-5mm}
\begin{longtable}{%
>{\sffamily\bfseries\itshape\small}p{0.26\textwidth}%
>{\small}p{0.68\textwidth}%
}%
\label{tabular:fields}\\
\caption{Команды заполнения полей документа}\\
\hline\hline
\multicolumn{1}{c}{\sffamily\bfseries{}Команда} &
\multicolumn{1}{c}{\sffamily\bfseries{}Описание}\\
\hline\hline
\endfirsthead
\caption{Команды заполнения полей документа. Продолжение}\\
\hline\hline
\multicolumn{1}{c}{\sffamily\bfseries{}Команда} &
\multicolumn{1}{c}{\sffamily\bfseries{}Описание}\\
\hline\hline
\endhead
\cellcolor{codecolor}%
\textbackslash{}AuthorSet\{<name>\} &
Печатает аргумент \sfemph{<name>} в поле \colorbox{resultcolor}{\sfemph{Разраб.}}
основной надписи.\\
\hline
\cellcolor{codecolor}%
\textbackslash{}CheckerSet\{<name>\} &
Печатает аргумент \sfemph{<name>} в поле \colorbox{resultcolor}{\sfemph{Пров.}} основной
надписи.\\
\hline
\cellcolor{codecolor}

\vspace{-4mm}
\textbackslash{}ScaleSet\{<value>\} &
Печатает аргумент \sfemph{<value>} в поле \colorbox{resultcolor}{\sfemph{Масштаб}}
основной надписи сборочного чертежа или чертежа печатной платы.\\
\hline
\cellcolor{codecolor}%
\textbackslash{}NormControllerSet \{<name>\} &
Печатает аргумент \sfemph{<name>} в поле \colorbox{resultcolor}{\sfemph{Н.~контр.}}
основной надписи.\\
\hline
\cellcolor{codecolor}%
\textbackslash{}TechControllerSet \{<name>\} &
Печатает аргумент \sfemph{<name>} в поле \colorbox{resultcolor}{\sfemph{Т.~контр.}}
основной надписи сборочного чертежа или чертежа печатной платы.\\
\hline
\cellcolor{codecolor}%
\textbackslash{}ApproverSet \{<name>\} &
Печатает аргумент \sfemph{<name>} в поле \colorbox{resultcolor}{\sfemph{Утв.}} основной
надписи.\\
\hline
\cellcolor{codecolor}

\vspace{1mm}
\textbackslash{}NameSet\{<name>\} &
Печатает аргумент \sfemph{<name>} в поле наименования изделия
основной надписи. Аргумент \sfemph{<name>} может быть как
однострочным, так и двустрочным. Разделение аргумента на строки производится с помощью
команды \bfemph{\textbackslash\textbackslash}. Например:\\\\[-4mm]
\multicolumn{2}{c}{%
\pcbdocmanualcode{%
\textcolor{Blue}{\textbackslash{}NameSet\{}Модуль%
\textcolor{Blue}{\textbackslash\textbackslash{}}расширителя сознания%
\textcolor{Blue}{\}}%
}}\\
\hline
\cellcolor{codecolor}

\vspace{3mm}
\textbackslash{}NumberSet \{<number>\} &
Печатает децимальный номер \sfemph{<number>} в поле обозначения
документа основной надписи схемы электрической принципиальной, перечня элементов,
сборочного чертежа и спецификации, а также в поле
\colorbox{resultcolor}{\sfemph{Перв. примен.}} схемы электрической принципиальной,
перечня элементов, чертежа печатной платы и сборочного чертежа.\\
\hline
\cellcolor{codecolor}%
\textbackslash{}PcbNumberSet \{<number>\} &
Печатает децимальный номер \sfemph{<number>} в поле обозначения
документа основной надписи чертежа печатной платы.\\
\hline
\cellcolor{codecolor}

\vspace{1mm}
\textbackslash{}PcbMaterialSet \{<name>\} &
Печатает аргумент \sfemph{<name>} в поле обозначения материала
детали основной надписи чертежа печатной платы. Аргумент
\sfemph{<name>} может быть однострочным, двустрочным или
трёхстрочным. Разделение аргумента на строки производится с помощью команды
\bfemph{\textbackslash\textbackslash}. Например:\\\\[-4mm]
\multicolumn{2}{c}{%
\pcbdocmanualcode{%
\textcolor{Blue}{\textbackslash{}PcbMaterialSet\{}Материал фольгированный%
\textcolor{Blue}{\textbackslash\textbackslash}\\качественный%
\textcolor{Blue}{\textbackslash\textbackslash{}}от надёжного
поставщика\textcolor{Blue}{\}}}}\\
\hline
\cellcolor{codecolor}%
\textbackslash{}PrimaryUseSet \{<number>\} &
Печатает аргумент \sfemph{<number>} в поле
\colorbox{resultcolor}{\sfemph{Перв. примен.}} основной надписи спецификации.\\
\hline\hline
\end{longtable}
