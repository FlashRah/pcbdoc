
\section{Шрифты}

По умолчанию в \emph{pcbdoc} используется наклонный шрифт \bfemph{GOST type A},
размер которого зависит от контекста. В качестве прямого имеется возможность
использовать шрифт \bfemph{GOST Type AU}. Локально сменить наклонный шрифт на прямой
и(или) изменить размер шрифта можно с помощью команд, приведённых в
\bfemph{Таблице~\ref{tabular:font}} и \bfemph{Таблице~\ref{tabular:fontit}}.~

\begin{tikztable}
{\topcaption{Команды изменения размера прямого шрифта}\label{tabular:font}}
\matrix
[
  %execute at end cell=\node{+};,
  %execute at empty cell=\node{+};,
  draw,
  nodes=
  {
    %draw,
    text depth=0.5mm,
    text height=3mm
  },
  column sep=3mm,
  row 1/.style=
  {
    font=\sffamily\bfseries,align=center
  },
  column 1/.style=
  {
    font=\sffamily\bfseries\itshape\small,
    align=left,
    text width=21mm
  },
  column 2/.style={align=left,text width=70mm},
  column 3/.style={align=left,text width=20mm}
]
{
  Команда & Пример использования & Результат\\
  |[fill=codecolor]|\textbackslash{}small       & |[fill=codecolor]|  &
    |[fill=resultcolor]|\smallresult{}Пупкин\\
  |[fill=codecolor]|\textbackslash{}normalfont  & |[fill=codecolor]|  &
    |[fill=resultcolor]|\normalfontresult{}Пупкин\\
  |[fill=codecolor]|\textbackslash{}llarge      & |[fill=codecolor]|  &
    |[fill=resultcolor]|\llargeresult{}Пупкин\\
  |[fill=codecolor]|\textbackslash{}large       & |[fill=codecolor]|  &
    |[fill=resultcolor]|\largeresult{}Пупкин\\
  |[fill=codecolor]|\textbackslash{}LLarge      & |[fill=codecolor]|  &
    |[fill=resultcolor]|\LLargeresult{}Пупкин\\
  |[fill=codecolor]|\textbackslash{}Large       & |[fill=codecolor]|  &
    |[fill=resultcolor]|\Largeresult{}Пупкин\\
};
\node[anchor=west] at(tikztable-2-2.179){%
\begin{pcbdoccode1}%
\AuthorSet{\small{}Пупкин}
\end{pcbdoccode1}
};
\node[anchor=west] at(tikztable-3-2.179){%
\begin{pcbdoccode1}%
\AuthorSet{\normalfont{}Пупкин}
\end{pcbdoccode1}
};
\node[anchor=west] at(tikztable-4-2.179){%
\begin{pcbdoccode1}%
\AuthorSet{\llarge{}Пупкин}
\end{pcbdoccode1}
};
\node[anchor=west] at(tikztable-5-2.179){%
\begin{pcbdoccode1}%
\AuthorSet{\large{}Пупкин}
\end{pcbdoccode1}
};
\node[anchor=west] at(tikztable-6-2.179){%
\begin{pcbdoccode1}%
\AuthorSet{\LLarge{}Пупкин}
\end{pcbdoccode1}
};
\node[anchor=west] at(tikztable-7-2.179){%
\begin{pcbdoccode1}%
\AuthorSet{\Large{}Пупкин}
\end{pcbdoccode1}
};
\draw[line width=0.6mm] (tikztable-1-1.north west) -- (tikztable-1-3.north east);
\end{tikztable}

\begin{tikztable}
{\topcaption{Команды изменения размера наклонного шрифта}\label{tabular:fontit}}
\matrix
[
  %execute at end cell=\node{+};,
  %execute at empty cell=\node{+};,
  draw,
  nodes=
  {
    %draw,
    text depth=0.5mm,
    text height=3mm
  },
  column sep=3mm,
  row 1/.style=
  {
    font=\sffamily\bfseries,align=center
  },
  column 1/.style=
  {
    font=\sffamily\bfseries\itshape\small,
    align=left,
    text width=23mm
  },
  column 2/.style={align=left,text width=75mm},
  column 3/.style={align=left,text width=20mm}
]
{
  Команда & Пример использования & Результат\\
  |[fill=codecolor]|\textbackslash{}smallit       & |[fill=codecolor]|  &
    |[fill=resultcolor]|\smallitresult{}Пупкин\\
  |[fill=codecolor]|\textbackslash{}normalfontit  & |[fill=codecolor]|  &
    |[fill=resultcolor]|\normalfontitresult{}Пупкин\\
  |[fill=codecolor]|\textbackslash{}llargeit      & |[fill=codecolor]|  &
    |[fill=resultcolor]|\llargeitresult{}Пупкин\\
  |[fill=codecolor]|\textbackslash{}largeit       & |[fill=codecolor]|  &
    |[fill=resultcolor]|\largeitresult{}Пупкин\\
  |[fill=codecolor]|\textbackslash{}LLargeit      & |[fill=codecolor]|  &
    |[fill=resultcolor]|\LLargeitresult{}Пупкин\\
  |[fill=codecolor]|\textbackslash{}Largeit       & |[fill=codecolor]|  &
    |[fill=resultcolor]|\Largeitresult{}Пупкин\\
};
\node[anchor=west] at(tikztable-2-2.179){%
\begin{pcbdoccode1}%
\AuthorSet{\smallit{}Пупкин}
\end{pcbdoccode1}
};
\node[anchor=west] at(tikztable-3-2.179){%
\begin{pcbdoccode1}%
\AuthorSet{\normalfontit{}Пупкин}
\end{pcbdoccode1}
};
\node[anchor=west] at(tikztable-4-2.179){%
\begin{pcbdoccode1}%
\AuthorSet{\llargeit{}Пупкин}
\end{pcbdoccode1}
};
\node[anchor=west] at(tikztable-5-2.179){%
\begin{pcbdoccode1}%
\AuthorSet{\largeit{}Пупкин}
\end{pcbdoccode1}
};
\node[anchor=west] at(tikztable-6-2.179){%
\begin{pcbdoccode1}%
\AuthorSet{\LLargeit{}Пупкин}
\end{pcbdoccode1}
};
\node[anchor=west] at(tikztable-7-2.179){%
\begin{pcbdoccode1}%
\AuthorSet{\Largeit{}Пупкин}
\end{pcbdoccode1}
};
\draw[line width=0.6mm] (tikztable-1-1.north west) -- (tikztable-1-3.north east);
\end{tikztable}

Кроме того, для указания типа шрифта и его размера можно воспользоваться встроенными
средствами \XeLaTeX{}. Например:

\begin{pcbdoccode}
\AuthorSet{\fontspec[Scale=0.68]{GOST type A}\itshape{}Пупкин}
\end{pcbdoccode}

Команда \bfemph{\textbackslash{}plusminus} печатает символ \symbol{177}. Например:

\begin{pcbdoccode}
\Element{RMC\_0805\_1\_KOM\plusminus{}5\%}{\refbox{R1}}{1}
\end{pcbdoccode}
