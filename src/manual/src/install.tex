
\section{Установка}

Класс \sfemph{pcbdoc} разрабатывался и использовался на компьютере с операционной
системой \sfemph{Debian GNU/Linux}. Вероятнее всего, он также будет работать и на
\sfemph{Windows}, и на \sfemph{MacOS}.

На машине пользователя \sfemph{pcbdoc} должен быть установлен и настроен
дистрибутив \sfemph{TeX~Live}. Процесс установки и настройки \sfemph{Tex~Live}
для конкретной платформы описан в его официальной документации.

Для установки \sfemph{pcbdoc} нужно скопировать дерево исходных файлов класса в
директорию, на которую указывает переменная \texttt{TEXMFHOME}, и установить
используемые в \sfemph{pcbdoc} шрифты для конкретной операционной системы.
\texttt{TEXMFHOME} является переменной дистрибутива \sfemph{Tex~Live},
указывающей на дерево, которое пользователи \sfemph{Tex~Live} могут использовать
для установки собственных пакетов, шрифтов или обновлённых версий системных
пакетов. По умолчанию оно находится в домашней директории, своей для каждого
пользователя.

Архив с классом \sfemph{pcbdoc} содержит две директории, \bfemph{texmf} и
\bfemph{.fonts}. В директории \bfemph{texmf} находится директория \bfemph{tex}, дерево
исходных файлов \sfemph{pcbdoc}. В директории \bfemph{.fonts} находятся файлы
используемых в \sfemph{pcbdoc} шрифтов.

Далее подразумевается, что на машине пользователя установлена операционная
система \sfemph{Debian GNU/Linux}, а командной оболочкой является \sfemph{bash}.

По умолчанию переменная \texttt{TEXMFHOME} указывает на директорию \bfemph{texmf} в
домашнем каталоге пользователя. В этом можно убедиться, выполнив команду

\pcbdocmanualterminal{
tlmgr conf | grep TEXMFHOME
}

Таким образом, в простейшем случае установка \sfemph{pcbdoc} сводится к извлечению
содержимого архива в домашний каталог пользователя:

\pcbdocmanualterminal{%
unzip pcbdoc.zip -d ~
}
