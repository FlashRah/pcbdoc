
\section{Перечень элементов}

Исходный код перечня элементов, как и любого другого документа \LaTeX{}, должен
начинаться с команды \bfsf{\textbackslash{}documentclass}. В основном аргументе команды%
(в фигурных скобках) следует указать класс документа \sfemph{pcbdoc}, а в
необязательном(в квадратных скобках) --- ключ \sfemph{doctype=pe}.

\pcbdocmanualcode{%
\ttfamily\textcolor{Blue}{\textbackslash{}documentclass[}%
doctype=pe\textcolor{Blue}{]\{}pcbdoc%
\textcolor{Blue}{\}}%
}

Далее(в преамбуле) должны находиться команды заполнения полей документа. Их можно
разместить либо непосредственно в преамбуле, либо в отдельном пакете, включаемом в
преамбулу с помощью команды \bfsf{\textbackslash{}usepackage}. Тело документа,
находящееся после преамбулы, должно начинаться с команды

\pcbdocmanualcode{%
\ttfamily\textcolor{Blue}{\textbackslash{}begin\{}document%
\textcolor{Blue}{\}}%
}

а заканчиваться командой

\pcbdocmanualcode{%
\ttfamily\textcolor{Blue}{\textbackslash{}end\{}document\textcolor{Blue}{\}}%
}

Внутри тела документа должно находиться
окружение \sfemph{ElementList}, которое начинается с команды

\pcbdocmanualcode{%
\ttfamily\textcolor{Blue}{\textbackslash{}begin\{}ElementList\textcolor{Blue}{\}}%
}

а заканчивается командой

\pcbdocmanualcode{%
\ttfamily\textcolor{Blue}{\textbackslash{}end\{}ElementList\textcolor{Blue}{\}}%
}

Внутри окружения \sfemph{ElementList} должны находиться команды заполнения строк
перечня элементов, приведённые в \bfsf{Таблице~\ref{tabular:pelines}}.\newpage

\begin{longtable}{%
>{\sffamily\bfseries\itshape\small}p{0.26\textwidth}%
>{\small}p{0.68\textwidth}%
}%
\label{tabular:pelines}\\
\caption{Команды заполнения строк перечня элементов}\\
\hline\hline
\multicolumn{1}{c}{\sffamily\bfseries{}Команда} &
\multicolumn{1}{c}{\sffamily\bfseries{}Описание}\\
\hline\hline
\endfirsthead
\caption{Команды заполнения строк перечня элементов. Продолжение}\\
\hline\hline
\multicolumn{1}{c}{\sffamily\bfseries{}Команда} &
\multicolumn{1}{c}{\sffamily\bfseries{}Описание}\\
\hline\hline
\endhead
\cellcolor{codecolor}

\vspace{-4mm}
\textbackslash{}Part\{<name>\} &
Печатает подчёркнутый аргумент \sfemph{<name>} в центре колонки
\colorbox{resultcolor}{\sfemph{Наименование}} Например:\\\\[-4mm]
\multicolumn{2}{c}{%
\pcbdocmanualcode{%
\textcolor{Blue}{\textbackslash{}Part\{}Микросхемы\textcolor{Blue}{\}}
}}\\
\hline
\cellcolor{codecolor}

\vspace{25mm}
\textbackslash{}Element[<note>]
\{<naming>\}

\{<refdes1 ... refdesN>\}

\{<quantity>\} &
Заполняет строку перечня элементов. Необязательный аргумент \sfemph{<note>} печатается
в колонке \colorbox{resultcolor}{\sfemph{Примечание}}. Аргументы \sfemph{<naming>},
\sfemph{<refdes1 ... refdesN>} и \sfemph{<quantity>} печатаются в колонках
\colorbox{resultcolor}{\sfemph{Наименование}},
\colorbox{resultcolor}{\sfemph{Поз. обозначение}} и
\colorbox{resultcolor}{\sfemph{Кол.}} соответственно.
Каждая запись в аргументе \sfemph{<refdes1 ... refdesN>} печатается в отдельной строке
колонки \colorbox{resultcolor}{\sfemph{Поз. обозначение}}. Каждую строку в колонке
\colorbox{resultcolor}{\sfemph{Поз. обозначение}} следует отделять от предыдущей одним
или несколькими пробельными символами(символ перехода на другую строку также является
пробельным символом). Между символом \bfsf{\{} и первой строкой, а также между последней
строкой и символом \bfsf{\}} пробельных символов быть не должно, поэтому в примере ниже
используется символ подавления последующих пробельных символов \bfsf{\%}. Каждую запись
в строке позиционных обозначений рекомендуется размещать в аргументе команды
\bfsf{\textbackslash{}refbox}, которая центрирует запись внутри колонки. Шрифт любой
записи строки позиционных обозначений можно немного уменьшить с помощью команды
\bfsf{\textbackslash{}llargeit}, что позволяет разместить в колонке чуть выступающий за
её пределы текст. Например:\\\\[-4mm]
\multicolumn{2}{c}{%
\pcbdocmanualcode{%
\textcolor{Blue}{\textbackslash{}Element\{}Y5V\textbackslash\_1206\textbackslash\_%
  4,7\textbackslash\_MKF\textbackslash\_20\textbackslash\%\textbackslash\_25V%
\textcolor{Blue}{\}\{\%}\\
~~\textcolor{Blue}{\textbackslash{}refbox\{}C6,C15,C16\textcolor{Blue}{\}}\\
~~\textcolor{Blue}{\textbackslash{}refbox\{}C18,C21\textcolor{Blue}{\}}\\
~~\textcolor{Blue}{\textbackslash{}refbox\{}C174,C175\textcolor{Blue}{\}}\\
~~\textcolor{Blue}{\textbackslash{}refbox\{}C180,C181\textcolor{Blue}{\}}\\
~~\textcolor{Blue}{\textbackslash{}refbox\{}C184-C187\textcolor{Blue}{\}}\\
~~\textcolor{Blue}{\textbackslash{}refbox\{}C190,C191\textcolor{Blue}{\}}\\
~~\textcolor{Blue}{\textbackslash{}refbox\{}C195-C199\textcolor{Blue}{\}}\\
~~\textcolor{Blue}{\textbackslash{}refbox\{}C201,C204\textcolor{Blue}{\}}\\
~~\textcolor{Blue}{\textbackslash{}refbox\{}C205\textcolor{Blue}{\}}\\
~~\textcolor{Blue}{\textbackslash{}refbox\{}\textbackslash{}llargeit%
  \textcolor{Blue}{\{\}}C207-C210\textcolor{Blue}{\}}\\
~~\textcolor{Blue}{\textbackslash{}refbox\{}C212\textcolor{Blue}{\}}\\
~~\textcolor{Blue}{\textbackslash{}refbox\{}\textbackslash{}llargeit%
  \textcolor{Blue}{\{\}}C232-C234\textcolor{Blue}{\}}\\
~~\textcolor{Blue}{\textbackslash{}refbox\{}\textbackslash{}llargeit%
  \textcolor{Blue}{\{\}}C238-C240\textcolor{Blue}{\}}\\
~~\textcolor{Blue}{\textbackslash{}refbox\{}\textbackslash{}llargeit%
  \textcolor{Blue}{\{\}}C265-C266\textcolor{Blue}{\}\%}\\
~~\textcolor{Blue}{\}\{}36\textcolor{Blue}{\}}
}}\\
\hline
\cellcolor{codecolor}\textbackslash\textbackslash & Переход на новую строку \\
\hline
\cellcolor{codecolor}\textbackslash{}newpage & Переход на новую страницу \\
\hline\hline
\end{longtable}
