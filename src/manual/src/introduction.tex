
\section{Введение}

Класс \emph{pcbdoc} предоставляет разработчикам печатных плат возможность
вёрстки конструкторской документации с помощью юникодной версии текстового
процессора \LaTeX{} --- \XeLaTeX{}. Подразумевается, что пользователь
имеет понятие о \TeX{} и способен сверстать в \LaTeX{} простейший
документ. С помощью \emph{pcbdoc} возможна вёрстка перечней элементов,
спецификаций, чертежей печатных плат, сборочных чертежей и схем электрических
принципиальных. Для вёрстки перечней элементов и спецификаций высокая
квалификация пользователя \LaTeX{} не требуется. Не требуется она и для
вёрстки чертежей печатных плат, сборочных чертежей и схем электрических
принципиальных, если рисунки, полученные из системы разработки печатных плат, не
требуют доработки. В этом случае необходимо просто вставить заранее
подготовленный рисунок в тело документа \emph{pcbdoc}. В противном случае
требуется знакомство пользователя с пакетом \emph{tikz}.

Данный класс был написан для личных целей, когда понадобилось оформить
конструкторскую документацию на уже разработанные печатные платы и сдать твёрдые
копии документов в архив. При разработке \emph{pcbdoc} ставилась задача
получения таких выходных документов, которые позволили бы беспрепятственно
пройти нормоконтроль на конкретном предприятии. Хотя полное соответствие ЕСКД и
не являлось главной целью \emph{pcbdoc}, \emph{сознательные} нарушения
стандартов при разработке класса не допускались. Автор надеется, что данный
класс пригодится кому-то ещё. В случае, если Вы найдёте \emph{pcbdoc}
полезным, но в чём-то не соответствующим ЕСКД или не позволяющим пройти Ваш
нормоконтроль, пожалуйста, сообщите об этом автору. Возможно, вместе мы исправим
это недоразумение.
