
\section{Опции и ключи класса pcbdoc}

Опцией класса \sfemph{pcbdoc} является идентификатор, влияющий на параметры вёрстки
документа. Ключом класса \sfemph{pcbdoc} является опция, имеющая некоторое
значение. Ключ представляет собой конструкцию типа \sfemph{<name>=<value>},
где \sfemph{<name>} является именем ключа, а \sfemph{<value>} -- его
значением. Ключи и опции указываются в необязательном аргументе команды
\bfsf{\textbackslash{documentclass}} и отделяются друг от друга запятыми.

\subsection{Тип документа}

Ключ \bfsf{doctype} определяет тип документа:

\begin{tabular}{cll}
  & \bfsf{pe~~} & Перечень элементов\\
  & \bfsf{pp~~} & Чертёж печатной платы\\
  & \bfsf{sb~~} & Сборочный чертёж\\
  & \bfsf{sch~} & Схема электрическая принципиальная\\
  & \bfsf{spec} & Спецификация
\end{tabular}

Если ключ \bfsf{doctype} не указан, типом документа по умолчанию является
спецификация.

\subsection{Размер страницы}

Если типом документа является чертёж печатной платы, сборочный чертёж или схема
электрическая принципиальная, с помощью ключа \bfsf{papersize} имеется
возможность указать размер страницы. Если же типом документа является перечень
элементов или спецификация, ключ \bfsf{papersize} игнорируется, и размер страницы
устанавливается в значение по умолчанию. Если ключ \bfsf{papersize} не указан,
размер страницы также устанавливается в значение по умолчанию.

Ключ \bfsf{papersize} может принимать значения \bfsf{a4}, \bfsf{a3}, \bfsf{a2},
\bfsf{a1}, \bfsf{a4x3} и \bfsf{a4x4}. Размером страницы по умолчанию является \bfsf{a4}.

\subsection{Толщины линий}

Ключи \bfsf{linethick} и \bfsf{linethin} задают ширину толстой и
тонкой линии соответственно. Значениями по умолчанию данных ключей являются
\bfsf{0.6mm} и \bfsf{0.3mm} соответственно.

\subsection{Лист изменений}

Если типом документа является спецификация или перечень элементов, опция
\bfsf{changelist} задаёт печать листа изменений в конце документа.

\subsection{Прочие опции и ключи}

По умолчанию \emph{pcbdoc} печатает пустую строку после каждой записи в перечне
элементов. При указании опции \bfsf{compactmode} вышеуказанная пустая строка
подавляется.

Опция \bfsf{draftmode} может использоваться при вёрстке сборочного чертежа, схемы
электрической принципиальной или чертежа печатной платы в \emph{черновом режиме}. В
черновом режиме вёрстки изменяется цвет фона и наносится координатная сетка, что
облегчает применение специфичных для данного типа документа команд. С помощью ключа
\bfsf{gridstep} в этом случае имеется возможность указать шаг сетки. Ключ
\bfsf{gridstep} по умолчанию имеет значение \bfsf{10mm}.
