
\section{Опции и ключи класса pcbdoc}

Опцией класса \emph{pcbdoc} является идентификатор, влияющий на параметры вёрстки
документа. Ключом класса \emph{pcbdoc} является опция, имеющая некоторое
значение. Ключ представляет собой конструкцию типа \coloremph{<name>=<value>}, где
\coloremph{<name>} является именем ключа, а \coloremph{<value>} -- его значением. Ключи
и опции указываются в необязательном аргументе команды
{\coloremph{\textbackslash{documentclass}}} и отделяются друг от друга запятыми.

\subsection{Тип документа}

Ключ \coloremph{doctype} определяет тип документа:

\begin{tabular}{cll}
  & \coloremph{pe~~} & Перечень элементов\\
  & \coloremph{pp~~} & Чертёж печатной платы\\
  & \coloremph{sb~~} & Сборочный чертёж\\
  & \coloremph{sch~} & Схема электрическая принципиальная\\
  & \coloremph{spec} & Спецификация
\end{tabular}

Если ключ \coloremph{doctype} не указан, типом документа по умолчанию является
спецификация.

\subsection{Размер страницы}

Если типом документа является чертёж печатной платы, сборочный чертёж или схема
электрическая принципиальная, с помощью ключа \coloremph{papersize} имеется
возможность указать размер страницы. Если же типом документа является перечень
элементов или спецификация, ключ \coloremph{papersize} игнорируется, и размер страницы
устанавливается в значение по умолчанию. Если ключ \coloremph{papersize} не указан,
размер страницы также устанавливается в значение по умолчанию.

Ключ \coloremph{papersize} может принимать значения \coloremph{a4},
\coloremph{a3}, \coloremph{a2}, \coloremph{a1}, \coloremph{a4x3} и
\coloremph{a4x4}. Размером страницы по умолчанию является \coloremph{a4}.

\subsection{Толщины линий}

Ключи \coloremph{linethick} и \coloremph{linethin} задают ширину толстой и
тонкой линии соответственно. Значениями по умолчанию данных ключей являются
\coloremph{0.6mm} и \coloremph{0.3mm} соответственно.

\subsection{Лист изменений}

Если типом документа является спецификация, опция \coloremph{changelist} задаёт
печать листа изменений в конце документа.

\subsection{Прочие опции и ключи}

По умолчанию \emph{pcbdoc} печатает пустую строку после каждой записи в перечне
элементов. При указании опции \coloremph{compactmode} вышеуказанная пустая строка
подавляется.

Опция \coloremph{draftmode} может использоваться при вёрстке сборочного чертежа, схемы
электрической принципиальной или чертежа печатной платы в \emph{черновом режиме}. В
черновом режиме вёрстки изменяется цвет фона и наносится координатная сетка, что
облегчает применение специфичных для данного типа документа команд. С помощью ключа
\coloremph{gridstep} в этом случае имеется возможность указать шаг сетки. Ключ
\coloremph{gridstep} по умолчанию имеет значение \coloremph{10mm}.
