
\section{Шрифты}

По умолчанию в \emph{pcbdoc} используется наклонный шрифт \coloremph{GOST type A},
размер которого зависит от контекста. В качестве прямого имеется возможность
использовать шрифт \coloremph{GOST Type AU}. Локально сменить наклонный шрифт на прямой
и(или) изменить размер шрифта можно с помощью команд, приведённых в
таблице~\ref{tabular:font} и таблице~\ref{tabular:fontit}.

\begin{longtable}{%
>{\cellcolor{codecolor}\ttfamily\bfseries}lc%
>{\cellcolor{codecolor}\ttfamily}lc%
>{\cellcolor{resultcolor}}l%
}%
\label{tabular:font}\\
\caption{Команды изменения размера прямого шрифта}\\
\hline\hline
\multicolumn{1}{c}{\sffamily\bfseries{}Команда} & &
\multicolumn{1}{c}{\sffamily\bfseries{}Пример использования} & &
\multicolumn{1}{c}{\sffamily\bfseries{}Результат}\\
%\hline%
\endfirsthead
\textbackslash{}small & &
\textbackslash{}AuthorSet\{\textbackslash{}small\{\}Пупкин\} & &
\smallresult{}Пупкин\\
%\hline
\textbackslash{}normalfont & &
\textbackslash{}AuthorSet\{\textbackslash{}normalfont\{\}Пупкин\} & &
\normalfontresult{}Пупкин\\
%\hline
\textbackslash{}llarge & &
\textbackslash{}AuthorSet\{\textbackslash{}llarge\{\}Пупкин\} & &
\llargeresult{}Пупкин\\
%\hline
\textbackslash{}large & &
\textbackslash{}AuthorSet\{\textbackslash{}large\{\}Пупкин\} & &
\largeresult{}Пупкин\\
%\hline
\textbackslash{}LLarge & &
\textbackslash{}AuthorSet\{\textbackslash{}LLarge\{\}Пупкин\} & &
\LLargeresult{}Пупкин\\
%\hline
\textbackslash{}Large & &
\textbackslash{}AuthorSet\{\textbackslash{}Large\{\}Пупкин\} & &
\Largeresult{}Пупкин\\
%\hline\hline
\end{longtable}

\begin{longtable}{%
>{\cellcolor{codecolor}\ttfamily\bfseries}lc%
>{\cellcolor{codecolor}\ttfamily}lc%
>{\cellcolor{resultcolor}\ttfamily}l%
}%
\label{tabular:fontit}\\
\caption{Команды изменения размера наклонного шрифта}\\
\hline\hline
\multicolumn{1}{c}{\sffamily\bfseries{}Команда} & &
\multicolumn{1}{c}{\sffamily\bfseries{}Пример использования} & &
\multicolumn{1}{c}{\sffamily\bfseries{}Результат}\\
%\hline%
\endfirsthead
\textbackslash{}smallit & &
\textbackslash{}AuthorSet\{\textbackslash{}smallit\{\}Пупкин\} & &
\smallitresult{}Пупкин\\
%\hline
\textbackslash{}normalfontit & &
\textbackslash{}AuthorSet\{\textbackslash{}normalfontit\{\}Пупкин\} & &
\normalfontitresult{}Пупкин\\
%\hline
\textbackslash{}llargeit & &
\textbackslash{}AuthorSet\{\textbackslash{}llargeit\{\}Пупкин\} & &
\llargeitresult{}Пупкин\\
%\hline
\textbackslash{}largeit & &
\textbackslash{}AuthorSet\{\textbackslash{}largeit\{\}Пупкин\} & &
\largeitresult{}Пупкин\\
%\hline
\textbackslash{}LLargeit & &
\textbackslash{}AuthorSet\{\textbackslash{}LLargeit\{\}Пупкин\} & &
\LLargeitresult{}Пупкин\\
%\hline
\textbackslash{}Largeit & &
\textbackslash{}AuthorSet\{\textbackslash{}Largeit\{\}Пупкин\} & &
\Largeitresult{}Пупкин\\
%\hline\hline
\end{longtable}

Кроме того, для указания типа шрифта и его размера можно воспользоваться встроенными
средствами \XeLaTeX{}. Например:

\pcbdocmanualcode{%
\textbackslash{}AuthorSet\{\textbackslash{}fontspec[Scale=0.68]\{GOST type A\}%
\textbackslash{}itshape\{\}Пупкин\}
}
